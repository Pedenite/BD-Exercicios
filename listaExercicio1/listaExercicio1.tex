%1) Definir: o que é registro e arquivo, o que é organização de arquivos, explicitar os tipos de organização de arquivos.
%2) Explique as diferenças entre arquivo sequencial e arquivo direto (acesso aleatório).
%3) Quais as operações básicas com arquivos?
%4) Explique, resumidamente e com suas palavras, como a linguagem Java permite a manipulação de arquivos.
%5) Defina os seguintes conceitos:
%5.1) sistema de informação (SI) e seus componentes;
%5.2) dado e banco de dados (BD);
%5.3) esquema de BD; instância de BD;
%5.4) SGBD; catálogo (ou dicionário) de um BD;
%5.5) sistema de bancos de dados (SBD) e seus componentes.
%6) Escrever um algoritmo para criar um arquivo contendo os dados dos produtos de uma loja. Os campos dos registros são: código do produto, descrição do produto e quantidade do produto em estoque. O algoritmo deve conter as operações: inclusão de produtos e listagem de todos os produtos. Após o algoritmo pronto, codifique-o usando a linguagem Java: faça uso de serialização/desserialização de objetos.
%7) Qual a diferença entre o papel do administrador de dados e do administrador de bancos de dados de uma organização? Explique as funções que cada papel deve exercer.
%8) Compare a abordagem de Sistemas de Bancos de Dados com a abordagem de Arquivos Tradicionais.
%9) Explique o papel de um SGBD no contexto de um Sistema de Bancos de Dados.
%10) Cite e caracterize pelo menos três papéis distintos de usuários de SGBDs.
%11) Que tipos de facilidades um SGBD deve prover para um software que deseja utilizar um banco de dados?
%12) Explique o conceito de redundância em BD. Caracterize a diferença entre redundância controlada e não-controlada em um SGBD.
%13) Defina os conceitos de integridade e de consistência em Bancos de Dados.
%14) Dê pelo menos dois exemplos de situações em que seria desaconselhável usar a abordagem de BD para suporte a um software?

\documentclass[12pt]{article}
\usepackage[brazil,american]{babel}
\usepackage[a4paper, total={6in, 8in}]{geometry}
\usepackage[utf8]{inputenc}

\usepackage{graphicx}
\usepackage{url}
\usepackage{float}
\usepackage{listings}
\usepackage{color}
\usepackage{todonotes}
\usepackage{algorithmic}
\usepackage{algorithm}
\usepackage{hyperref}
\usepackage{graphicx}

\begin{document}

\title{Lista de Exercícios 1 BD}
\author{Pedro Hennrique de Brito Agnes}

\maketitle

\section{Organização de Arquivos}
\textbf{Registro} é um conjunto de campos agrupado, que mantém um nível de organização mais alto. Se tratam de uma ferramenta lógica e não física que possuem tamanho fixo em geral. 

\noindent
\textbf{Arquivo} é um conjunto de dados que se relacionam de alguma forma, ou seja, juntos descrevem uma informação ou conjunto de informações. Podem ter diferentes formatos que proporcionam diversos usos como armazenar o código de um algoritmo, logs, informações processadas por um programa ou até fotos e vídeos.

\noindent
\textbf{Organização de arquivos} trata-se do método usado para dispor todos esses conjuntos de dados, visando uma melhor acessibilidade a esses dados.

\section{Arquivo sequencial e arquivo direto}
Em um arquivo sequencial, registros são distribuídos em uma certa ordem, um após o outro, dentro de uma área.
Já em um arquivo direto (acesso aleatório), ao invés de um índice é utilizada uma função (hashing) que calcula o endereço do registro a partir da chave do registro.

\section{Operações básicas com arquivos}
Dentre as operações básicas com arquivos, tem-se as de abertura, leitura, escrita e fechamento. A maioria das linguagens de programação apresentam suporte a essas operações em bibliotecas nativas.

\section{Manipulação de arquivos com a linguagem Java}
Uma forma que a linguagem java permite a manipulação de arquivos é usando um objeto do tipo FileReader/Writer para ler/escrever no arquivo a ser referenciado no construtor, sendo aberto durante a chamada dele.
Em seguida, o programador pode usar os vários métodos disponíveis para a manipulação desejada do arquivo e, por fim, chamar o método que fecha o arquivo quando finalizar as manipulações.

\section{Definição dos conceitos:}

\subsection{Sistema de informação (SI) e seus componentes}
Sistema de informação é um conjunto de componentes para coletar, armazenar e processar dados, além de disponibilizar informação, conhecimento e produtos digitais.
Um sistema de informação é essencialmente composto por \textit{hardware}, \textit{software}, banco de dados, rede e pessoas que se integram para executar entrada, saída, processo, controle e feedback.

\subsection{Dado e banco de dados (BD)}
Um dado é uma informação armazenada por um sistema, podendo ser composto de números, caracteres, ou até um conjunto desses vindo de um programa.

Banco de dados é um modo de persistir esses dados de forma a permitir um acesso a eles com compartilhamento facilitado, menos redundâncias e maior segurança.

\subsection{Esquema de BD e instância de BD}
O esquema de um banco de dados é uma representação de sua estrutura de acordo com o modelo de dados.
Já a instância seria como os dados estão organizados no banco em um certo instante.

\subsection{SGBD e catálogo de um BD}
Sistema de gerenciamento de banco de dados são ferramentas que permitem o acesso a todos os dados do banco de forma organizada, contendo tabelas, colunas, e geralmente dispõem de uma linguagem como o SQL para a manipulação dos dados,
que permite coisas como a inserção, seleção e deleção de dados em uma tabela, organizando os dados por coluna.
Já o catálogo (ou dicionário) de um BD é uma unidade que armazena os esquemas utilizados pelo banco mantido pelo SGBD.

\subsection{Sistema de bancos de dados (SBD) e seus componentes}
Um sistema de banco de dados é um sistema que permite abstração de dados isolando o usuário dos detalhes mais internos do banco e permite independência dos dados às aplicações.
Ele é essencialmente, um mantedor de registros e tem como componentes principais, os dados, o software, o hardware e o usuário.

\section{Algoritmo}
\begin{itemize}
\item Criar um algoritmo para criar um arquivo contendo os dados dos produtos de uma loja.
\item Os campos dos registros são: código do produto, descrição do produto e quantidade do produto em estoque.
\item O algoritmo deve conter as operações: inclusão de produtos e listagem de todos os produtos.
\item Deve-se usar a linguagem Java contendo a serialização/desserialização de objetos
\end{itemize}

\section{Administrador de BD x Administrador de dados}
Um administrador de banco de dados tem a função de instalação, manutenção e atualização do SGBD, tendo que verificar a forma mais eficiente e otimizada de atender aos requisitos.
Já um administrador de dados tem a função de manter atualizados os modelos de dados da organização e garantir a qualidade dos mesmos para atender às necessidades da empresa.

\section{Abordagem SBD x Arquivos Tradicionais}
Dentre as vantagens de se usar uma abordagem de sistema de banco de dados sobre uma abordagem de arquivos tradicional, temos a rapidez na manipulação e no acesso aos dados, redução de redundâncias e de inconsistência de informações, compartilhamento facilitado entre aplicações, maior segurança, menor esforço para o desenvolvimento das aplicações, entre outros.
Porém, como vantagens para a abordagem de arquivos tradicional, temos que é um padrão aberto, logo não precisa pagar por nenhum \textit{software}, muitas linguagens oferecem suporte amplo para a manipulação dearquivos, é mais simples tanto para o usuário quanto para o computador, permite a construção de aplicações mais fáceis, entre outros.

\section{Explique o papel de um SGBD no contexto de um Sistema de Bancos de Dados}
questão 9

\section{Cite e caracterize pelo menos três papéis distintos de usuários de SGBDs}
questão 10

\section{Que tipos de facilidades um SGBD deve prover para um software que deseja utilizar um banco de dados?}
questão 11

\section{Explique o conceito de redundância em BD. Caracterize a diferença entre redundância controlada e não-controlada em um SGBD}
questão 12

\section{Defina os conceitos de integridade e de consistência em Bancos de Dados}
questão 13

\section{Dê pelo menos dois exemplos de situações em que seria desaconselhável usar a abordagem de BD para suporte a um software?}
questão 14

\end{document}