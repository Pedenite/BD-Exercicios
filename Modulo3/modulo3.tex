%15) Que tipos de linguagens existem no contexto de Sistemas de Bancos de Dados (DDL, DML, etc.)? Quais os objetivos de cada linguagem?
%16) O que é um Modelo de Dados? Defina o objetivo de um Modelo de Dados.
%17) Explique o que é a atividade de Projeto de Banco de Dados. Caracterize os níveis de abstração envolvidos: descritivo, conceitual, operacional e físico.
%18) Descreva os principais componentes de um SGBD, explicitando sua funcionalidade. Desenhe um diagrama ilustrando as relações entre os componentes do SGBD.
%19) Explique a arquitetura ANSI/SPARC.
%20) O que é independência de dados? Dê exemplo de independência de dados física e lógica no contexto da arquitetura ANSI/SPARC de SGBD

\documentclass[12pt]{article}
\usepackage[brazil,american]{babel}
\usepackage[a4paper, total={6in, 8in}]{geometry}
\usepackage[utf8]{inputenc}

\usepackage{graphicx}
\usepackage{url}
\usepackage{float}
\usepackage{listings}
\usepackage{color}
\usepackage{algorithmic}
\usepackage{algorithm}
\usepackage{hyperref}
\usepackage{graphicx}

\begin{document}

\title{Tarefas do Módulo 3 BD}
\author{Pedro Henrique de Brito Agnes}

\maketitle

\setcounter{section}{14}
\section{Tipos de Linguagens SBD}
Os tipos de linguagem de um SBD são:
\begin{itemize}

\item
\textbf{DDL}, \textit{Data Definition Language} - Tem como objetivo, definir a estrutura dos dados e esquemas do BD.

Dentre os comandos DDL, temos \texttt{CREATE}, \texttt{ALTER} e \texttt{DROP}

\item
\textbf{DML}, \textit{Data Manipulation Language} - Tem como objetivo, acessar e manipular os dados das tabelas.

São comandos DML : \texttt{SELECT}, \texttt{INSERT}, \texttt{DELETE} e \texttt{UPDATE}

\item
\textbf{DTL}, \textit{Data Transaction Language} - Tem como objetivo, o controle de transação, que roda as mudanças feitas pelo DML.

São comandos DTL : \texttt{BEGIN TRANSACTION}, \texttt{COMMIT} e \texttt{ROLLBACK}

\item
\textbf{DCL}, \textit{Data Control Language} - Tem como objetivo, controlar a parte de segurança do banco de dados, retornando os dados salvos ou guardados.

São comandos DCL : \texttt{GRANT}, \texttt{REVOKE} e \texttt{DENY}

\item 
\textbf{SDL}, \textit{Storage Definition Language} - Tem como objetivo, especificar o esquema interno.

\item
\textbf{VDL}, \textit{View Definition Language} - Tem como objetivo, especificar a visão do usuário e seu mapeamento para um esquema conceitual.

\end{itemize}

\section{Modelo de Dados}
Um modelo de dados é o modelo para organização dos dados de um BD que pode ter diferentes tipos de abstração. É classificado como alto nível, nível médio e baixo nível.

O alto nível representa a forma como os dados são compreendidos pelos usuários. Já o nível médio é implementado pelo SGBD e é divido em vários tipos, como modelo relacional, hierárquico, orientado a objetos, entre outros. Por fim, o baixo nível, simplesmente descreve como os dados são armazenados fisicamente.

O principal objetivo de um modelo de dados é o de disponibilizar uma representação conceitual dos dados.

\section{Atividade de Projeto de BD}
Explique o que é a atividade de Projeto de Banco de Dados. Caracterize os níveis de abstração envolvidos: descritivo, conceitual, operacional e físico.

\section{Componentes de um SGBD}
Descreva os principais componentes de um SGBD, explicitando sua funcionalidade. Desenhe um diagrama ilustrando as relações entre os componentes do SGBD.

\section{Arquitetura ANSI/SPARC}
Explique a arquitetura ANSI/SPARC.

\section{Independência de dados}
O que é independência de dados? Dê exemplo de independência de dados física e lógica no contexto da arquitetura ANSI/SPARC de SGBD.

\end{document}